\[
    I_D=\frac{1}{2} \cdot K P_N \cdot \frac{W}{L} \cdot\left(U_{G S}-U_{T H}\right)^2
\]
\[
    \frac{W}{L}=\frac{2\cdot I_{D}}{KP_{N}\cdot (U_{GS} -U_{TH})^2 } 
\]
\[
    \frac{W_{1} }{L_{1} }=\frac{2\cdot \num{25e-6}}{\num{220e-6}\cdot (\num{0.2})^2 } 
\]
\[
    \frac{W_{1} }{L_{1} }\doteq \num[round-mode=places,round-precision=2]{5.681818181818182} 
\]
Pro oba tranzistory zvolíme stejnou délku kanálu \(L_{1}=L_{2} = \qty{0.2}{\micro\meter}\), tedy \(W_{1} =\qty[round-mode=places,round-precision=2]{1,136363636363636}{\micro\meter}\). Druhý tranzistor má dosáhnout dvakrát vyššího proudu, takže zvolíme \(W_{2} =\qty{2.28}{\micro\meter}\)  

\[
    R_1=\frac{U_R}{I_{M 1}}=\frac{U_{C C}-U_{G S 1}}{I_{M 1}}=\frac{U_{C C}-\left(U_{T H 0,1}+U_{O V, 1}\right)}{I_{M 1}}
\]
\[
    R_1=\frac{\num{1.8}-\left(\num{368,024e-3}+\num{0.2}\right)}{\num{25e-6}}
\]
\[
    R_{1} \doteq \qty{49,28}{\kilo\ohm}
\]


\[
    r_{O U T}=\frac{1}{\lambda \cdot I_{M 2}}
\]

\[
    r_{O U T}=\frac{1}{\num[round-mode=places,round-precision=3]{0,0437895} \cdot \num{50e-6}}
\]
\[
    r_{O U T}=\qty[round-mode=places,round-precision=3]{454,5454545454545}{\kilo\ohm}
\]



\subsubsection{Kaskodové}
Pro vstupní větev je stanoven proud \qty{50}{\micro\ampere}, výpočet rozměrů je proveden obdovným způsobem jako v minulém přkladu:
\[
    \frac{W_3}{L_3}=\frac{2\cdot I_{D3}}{KP_{P}\cdot (U_{GS} -U_{TH})^2 } 
\]
\[
    \frac{W_3}{L_3}=\frac{2\cdot \num{50e-6}}{\num{60e-6} \cdot (\num{0.2})^2 } 
\]
\[
    \frac{W_3}{L_3}=\num{41,67}
\]

Pro nastavení proudu slouží rezistor R2. Jeho hodnota je stanovena na základě úbytku napětí na rezistoru.
% U_bs pro M3 je U_GS1 coz je U_th0 + U_ov
% z tabulky pak odecitam radek pro nejvyssi U_bs, coz je porad mene...
% zaokrouhleno nahoru 600m
\begin{align*}
    U_{R 2}=&U_{C C}-U_{G S 1}-U_{G S 3} \\
           =&U_{C C}-U_{T H 0,1}- U_{O V 1}-U_{T H, 3}-U_{O V 3} \\
           =&U_{C C}-U_{T H 0,1}-U_{T H, 3}-2 \cdot U_{O V 1,3} \\
           =&\num{1.8}-\num{0.4433}-\num{0.6}-2 \cdot \num{0.2} \\
           =&\qty{0.3567}{\volt}
\end{align*}

\begin{align*}
    R_{2} =& \frac{U_{R1}}{I_{M1} } \\
          =& \frac{\num{0.3567}}{\num{50e-6}} \\
          =& \qty{7.134}{\kilo\ohm}
\end{align*}

\[
    r_{OUT} =r_{o2} \cdot r_{o4} \cdot g_{m4} 
\]
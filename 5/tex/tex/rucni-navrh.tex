Označení součástek v této kapitole odpovídá Obr.~\ref{fig:spice1.png}.

Nejprve zvolíme hodnotu kompenzační kapacity:
\[
    C_{C} = \num{0.3}\cdot C_{L} = \num{0.3}\cdot  \num{5e-12} = \qty{1.5}{pF}
\] 

Dále je potřeba stanovit minimální potřebné proudy v obvodu:
\begin{align*}
    GBW &= \frac{\frac{2\cdot I_{1} }{U_{OV1} }}{2\cdot \pi \cdot C_{C} } \\ 
    I_{1} &= U_{OV1}\cdot \pi \cdot C_{C} \cdot GBW \\ 
    I_{1} &= \num{0.2}\cdot \pi \cdot \num{1.5e-12} \cdot \num{10e6} \\ 
    I_{1} &= \qty{9.42}{\micro A}
\end{align*}

\begin{align*}
    SR_{int}  &= \frac{I_{p3}}{C_{C} } \\ 
    I_{p3} &= SR_{int}\cdot C_{C}  \\
    I_{p3} &= \num{5e6}\cdot \num{1.5e-12}  \\
    I_{p3} &= \qty{7.5}{\micro A}  \\
\end{align*}

    Aby bylo vyhověno všem parametrům a zachována jistá návrhová rezerva byl zvolen proud \(I_{p3} =\qty{20}{\micro A}\) a proudy \(I_{1}=I_{2}= \qty{10}{\micro A}  \).

    Ze stanovených proudů lze vypočítat rozměry tranzistorů:
    \begin{align*}
        \frac{W_{p3}}{L} &= \frac{2\cdot I_{p3}}{KP_{P}\cdot (U_{GS} -U_{TH})^2 } \\
        \frac{W_{p3}}{L} &= \frac{2\cdot \num{20e-6}}{\num{60e-6}\cdot \num{0.2}^2 } \\
        \frac{W_{p3}}{L} &= \num{16.67}
    \end{align*}
    Tedy \(W_{p3} = \qty{33.33}{\micro m}\). Proud touto větví se dále dělí na půl, tedy platí \(W_{p1,2} = W_{p3} /2 = \qty{16.67}{\micro m}\). 

    Ekvivalentní výpočet pro tranzistorů typu N:
    \begin{align*}
        \frac{W_{n1,2}}{L} &= \frac{2\cdot I_{n1,2}}{KP_{N}\cdot (U_{GS} -U_{TH})^2 } \\
        \frac{W_{n1,2}}{L} &= \frac{2\cdot \num{10e-6}}{\num{220e-6}\cdot \num{0.2}^2 } \\
        \frac{W_{n1,2}}{L} &= \num{2.27}
    \end{align*}
    Tedy \(W_{n1,2} = \qty{4.54}{\micro m}\).

    Z podmínky pro fázovou bezpěčnost alespoň \qty{60}{\degree} vyplývá pro druhý stupeň desetkrát větší proud než pro první stupeň. Tedy i rozměry tranzistorů ve výstupní větvi budou desetkrát větší, platí:
    \begin{align*}
        W_{p4} &= 10\cdot W_{p2} = \qty{166.7}{\micro m} \\
        W_{n3} &= 10\cdot W_{n2} = \qty{45.4}{\micro m} \\
    \end{align*}

    Pro poslední tranzistor \(M_{p5} \) zvolíme stejný proud (a tedy i rozměry) jako pro \(M_{p3} \), zbývá dopočíst hodnotu \(R_{1}\):
    \begin{align*}
        R_{1} &= \frac{U_{CC} - (U_{DSp5min} + U_{TH0p5})  }{I_{p5} } \\
        R_{1} &= \frac{\num{1.8} - (\num{0.2} + \num{0.43}) }{\num{20e-6} } \\
        R_{1} &= \qty{58.5}{k\ohm}
    \end{align*}

    

\subsubsection{Předpokládané hodnoty parametrů}
    Na základě zvolených hodnot proudů a rozměrů součástek je potřeba znovu přepočítat některé parametry:
    \begin{align*}
        GBW &= \frac{\frac{2\cdot I_{1} }{U_{OV1} }}{2\cdot \pi \cdot C_{C} } \\ 
        GBW &= \frac{\frac{2\cdot \num{10e-6} }{\num{0.2} }}{2\cdot \pi \cdot \num{1.5e-12} } \\ 
        GBW &= \qty{10.61}{MHz}
    \end{align*}

    \begin{align*}
        SR_{int}  &= \frac{I_{p3}}{C_{C} } \\ 
        SR_{int}  &= \frac{\num{20e-6}}{\num{1.5e-12}} \\ 
        SR_{int}  &= \qty{13.33}{V \per\micro s}
    \end{align*}

    Odhadovaná spotřeba zařízení:
    \begin{align*}
        P&=U_{CC}\cdot (I_{p5} +I_{p3} +I_{p4} ) \\
        P&=\num{1.8}\cdot (\num{20e-6} +\num{20e-6} +\num{100e-6} ) \\
        P&=\qty{252.00}{\micro W} \\
    \end{align*}

    Vstupní rozsah souhlasného napětí:
    \begin{align*}
        U_{ICMRmin} &= U_{TH,n}-U_{TH,p} +U_{OV3} \\ 
        U_{ICMRmin} &= \num{387.106e-3}-\num{443.3e-3} +\num{0.2} \\ 
        U_{ICMRmin} &= \qty{143,8}{mV}
    \end{align*}

    \begin{align*}
        U_{ICMRmax} &= U_{TH,p} + U_{OV1} + U_{OV5}  \\ 
        U_{ICMRmax} &= \num{443.3e-3} +\num{0.2}+\num{0.2} \\ 
        U_{ICMRmax} &= \qty{843.3}{mV}
    \end{align*}

    Výstupní napětový rozsah:
    \begin{align*}
        OVS &= U_{CC} -U_{OV7} -U_{OV6} \\ 
        OVS &= \num{1.8} -\num{0.2} -\num{0.2} \\
        OVS &= \qty{1.4}{V}
    \end{align*}

    Zesílení:
    \begin{align*}
    A_{U0} &= g_{m 1} \cdot\left(r_{D S 2} \| r_{D S 4}\right) \cdot g_{m 6} \cdot\left(r_{D S 6} \| r_{D S 7}\right) \\
    A_{U0} &= \frac{2\cdot I_{1}}{U_{OV1}} 
        \cdot\left(\frac{\frac{1}{\lambda_2 I_2} \cdot \frac{1}{\lambda_4 I_4}}{\frac{1}{\lambda_2 I_2}+\frac{1}{\lambda_4 I_4}}\right) \cdot \frac{2\cdot I_{6} }{U_{OV6} } 
        \cdot\left(\frac{\frac{1}{\lambda_6 I_6} \cdot \frac{1}{\lambda_7 I_7}}{\frac{1}{\lambda_6 I_6}+\frac{1}{\lambda_7 I_7}}\right) \\
    A_{U0} &= \frac{2\cdot \num{10e-6}}{\num{0.2}} 
        \cdot\left(\frac{\frac{1}{\num{0.08}\cdot  \num{10e-6}} \cdot \frac{1}{\num{0.04}\cdot  \num{10e-6}}}{\frac{1}{\num{0.08}\cdot  \num{10e-6}}+\frac{1}{\num{0.04}\cdot  \num{10e-6}}}\right) \cdot \frac{2\cdot \num{100e-6} }{\num{0.2} } 
        \cdot\left(\frac{\frac{1}{\num{0.04}\cdot  \num{100e-6}} \cdot \frac{1}{\num{0.08}\cdot  \num{100e-6}}}{\frac{1}{\num{0.04} \cdot \num{100e-6}}+\frac{1}{\num{0.08}\cdot  \num{100e-6}}}\right) \\
        A_{U0} &= \num{6944.44} = \qty{76.83}{dB}
    \end{align*}





\subsubsection{Zapojení na tranzistorové úrovni}
    \begin{figure}[h!]
        \centering
        \includegraphics[scale=0.5]{spice1.png}
        \caption{Vnitřní zapojení OTA zesilovače.}
        \label{fig:spice1.png}
    \end{figure}
    
    Z analýzy .OP lze vypočítat spotřebu zapojení:
    \begin{align*}
        P&=U_{CC}\cdot (I_{p5} +I_{p3} +I_{p4} ) \\
        P&=\num{1.8}\cdot (\num{20.34e-6} +\num{21.142e-6} +\num{106.07e-6} ) \\
        P&=\qty{265.59}{\micro W} \\
    \end{align*}
